\documentclass[a4paper]{article}

\def\nterm {Spring}
\def\nyear {2025}
\def\ncourse {Natural Language Processing}

\input{header}

\begin{document}
\maketitle

\tableofcontents

\section{Introduction}

Below is a list of topics that will be covered in this note. 
The topics include fundamental theoretical knowledge in 
language processing and language modeling in the first 
few sections, as well as frontier research results
in later sections.

\begin{enumerate}
  \item Language modeling fundamentals 
  \begin{enumerate}
    \item Representing words 
    \item Language modeling 
    \item Sequence modeling architectures
  \end{enumerate}

  \item Training and inference models
  \begin{enumerate}
    \item Decoding and Generation Algorithms
    \item In-context learning
    \item Pre-training
    \item Fine-tuning
    \item Reinforcement Learning
  \end{enumerate}

  \item Evaluation and Experimental Design
  \begin{enumerate}
    \item Evaluating Language Generators
    \item Experimental Design
    \item Human Annotation
    \item Debugging/Interpretation Techniques
  \end{enumerate}

  \item Advanced Algorithms and Architectures 
  \begin{enumerate}
    \item Advanced Pretraining, Post-Training, and Inference
    \item Retrieval and Retrieval-augmented Generation
    \item Long Sequence Models
    \item Distillation and Quantization
    \item Ensembling and Mixture of Experts
  \end{enumerate}

  \item NLP Applications and Society
  \begin{enumerate}
    \item Complex Reasoning Tasks
    \item Language Agents
    \item Multimodal NLP
    \item Multilingual NLP
    \item Bias and Fairness
  \end{enumerate}
\end{enumerate}

\section{Language Modeling Fundamentals}


\end{document}